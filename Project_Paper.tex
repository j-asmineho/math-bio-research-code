\documentclass[12pt]{article}
\usepackage{scrtime} % for \thistime (this package MUST be listed first!)
\usepackage[margin=1in]{geometry}
\usepackage{amsmath} % essential for cases environment
\usepackage{amsthm} % for theorems and proofs
\usepackage{amsfonts} % mathbb
\usepackage{graphics,graphicx}
\usepackage{lineno}
\usepackage{color}
\definecolor{aqua}{RGB}{0, 128, 225}
\usepackage[colorlinks=true,allcolors=blue]{hyperref}
\pagecolor{white}
\usepackage{float}
\setlength{\parindent}{15pt} 
%%%%%%%%%%%%%%%%%%%%%%%%%%%%%%%%%%%
%% FANCY HEADER AND FOOTER STUFF %%
%%%%%%%%%%%%%%%%%%%%%%%%%%%%%%%%%%%
\usepackage{fancyhdr,lastpage}
\pagestyle{fancy}
\fancyhf{} % clear all header and footer parameters
%%%\lhead{Student Name: \theblank{4cm}}

%%%\chead{}
%%%\rhead{Student Number: \theblank{3cm}}
%%%\lfoot{\small\bfseries\ifnum\thepage<\pageref{LastPage}{CONTINUED\\on next page}\else{LAST PAGE}\fi}
\lfoot{}
\cfoot{{\small\bfseries Page \thepage\ of \pageref{LastPage}}}
\rfoot{}
\renewcommand\headrulewidth{0pt} % Removes funny header line
%%%%%%%%%%%%%%%%%%%%%%%%%%%%%%%%%%%

\title{Measles Transmission Dynamics with Varying Vaccination Proportions}

\author{\underline{\emph{Group Name}}: \texttt{{\color{blue}Acute Triangle}}\\
{}\\
\underline{\emph{Group Members}}: {\color{blue}Sarah Leong, Nimer Boparai, Jasmine Ho}}

\date{\today\ @ \thistime}

\begin{document}

\maketitle

%%%%%%%%%%%%%%%%%%%%%%%%%%%%%%%%%%%
%% Abstract %%
%%%%%%%%%%%%%%%%%%%%%%%%%%%%%%%%%%%

\begin{abstract}
This paper explores the dynamics of measles transmission in relation to vaccination proportions and measles epidemic parameters, comparing it to New York's measles time series. Using a Susceptible-Infected-Removed (SIR) model, equilibrium points were determined, showing damped oscillations. A stochastic Gillespie simulation model was introduced to study the effects of varying vaccination rates, focusing on the transition from biennial to annual epidemic cycles. The analysis of 
\end{abstract}

\tableofcontents

%%%%%%%%%%%%%%%%%%%%%%%%%%%%%%%%%%%
%% Background %%
%%%%%%%%%%%%%%%%%%%%%%%%%%%%%%%%%%%
\section{Introduction}

% Enable paragraph indentation

\indent Measles is an airborne disease that can lead to severe health complications and death, especially in children. It is highly contagious and continues to be a significant public health issue, particularly in places such as Asia and sub-Saharan Africa, which account for over 95\% of global measles deaths and 15\% of under-five mortality (Tilahun, Demie, \& Eyob, 2020). In 2023, approximately 107,500 deaths were from measles, primarily among unvaccinated individuals and under-vaccinated children under five (World Health Organization, 2024). Although there are existing vaccines, global vaccination coverage faces challenges from factors such as accessibility and vaccine hesitancy, which contribute to persistent outbreaks even in developed countries.

\indent High vaccination coverage is the most effective method to prevent measles outbreaks, reduce mortality, and disrupt transmission cycles. The study of vaccination efficacy and its influence on disease dynamics is significant in epidemiology, as it not only reduces prevalence but also alters transmission dynamics, often reshaping periodic patterns observed in endemic diseases. This paper investigates how varying vaccination rates impact measles transmission dynamics, particularly analyzing the shift from biennial cycles to annual cycles following the introduction of the measles vaccine in 1963.

\indent Historically, between 1946 and 1963, measles in New York exhibited a biennial periodicity driven by seasonal transmission and the accumulation of susceptible individuals (Hempel \& Earn, 2015). These cycles are well-documented through time-series and mechanistic models, which highlight the role of birth rates in sustaining susceptible populations. Thus, the addition of vital dynamics to these models is crucial for accurately capturing these oscillations in this recurrent epidemic.

\indent The introduction of vaccinations in 1963 evidently disrupted this biennial cycle, triggering a transition to an annual cycle as vaccination coverage increased. This paper explores these observed changes by analyzing dynamics under varying vaccination rates, from partial to high coverage, using time-series and mechanistic modeling approaches. By quantifying how specific vaccination proportions alter periodicity and outbreak magnitude, this study offers valuable insights into the interplay between public health interventions, disease dynamics, and the potential to guide vaccination policies for sustainable disease control.


%%%%%%%%%%%%%%%%%%%%%%%%%%%%%%%%%%%
%% Methods%%
%%%%%%%%%%%%%%%%%%%%%%%%%%%%%%%%%%%
\section{Methods}

\subsection {Model Assumptions and Parameters}
For the analysis with the deterministic and stochastic model, the assumptions remain consistent. We assume that all individuals infected with the disease have the same probability of dying. For those who are susceptible, the time from infection to death is constant across individuals. The transmission of measles is through contact between susceptible and infected individuals, with the assumption that every individual has an equal likelihood of interaction. The assumption that death from disease is excluded.

\bigskip
\indent
The table below outlines the parameters used in modeling measles epidemics, where values are based on established epidemiological data. (Cite Earn?)
\begin{table}[h!]
  \begin{center}
    \renewcommand{\arraystretch}{1.5} 
    \begin{tabular}{l|c|r} 
      \textbf{Parameter} & \textbf{Value} & \textbf{Description}\\
      \hline
      $N$ & 5,000,000 & Total Population Size \\
      $R_0$ & 17 & Basic Reproduction Rate \\
      $\mu$ & $\frac{1}{50}$ & Annual Birth/Death Rate \\
      $\gamma$ & $\frac{365}{13}$ & Recovery Rate \\
      $\beta$ & $R_0 \cdot (\gamma + \mu)$ & Transmission Rate \\ 
    \end{tabular}
    
  \end{center}
\end{table}

\subsection {The SIR Model}
Considering the standard SIR model with vital dynamics: 
    \begin{align}
      \frac{dS}{dt} &= \mu N -\frac{\beta}{N} SI - \mu S\\
      \noalign{\vspace{8pt}}
      \frac{dI}{dt} &= \frac{\beta}{N} SI - \gamma I - \mu I\\
      \noalign{\vspace{8pt}}
      \frac{dR}{dt} &= \gamma I - \mu R
    \end{align}

\indent Where $S$, $I$ and $R$ denote the numbers of susceptible, infectious and removed individuals, respectively, and $N=S+I+R$ is the total population size. The SIR model is sufficient in this analysis because the latent period is not critical for the susceptible-infective interaction, allowing the exclusion of the exposed compartment. While the SEIR model is generally considered a more accurate representation of measles dynamics due to the relatively long mean latent period compared to the mean infectious period, research by Krylova and Earn (2015) demonstrated that the SIR model can closely approximate the SEIR model when the mean generation interval is used in place of the mean infectious period within the SIR model.

\begin{figure}[H]
  \centering
  \includegraphics[width=0.8\textwidth]{Images/Measles_Transmission_Model.png}
   \caption{SIR Diagram} 
\label{fig:example}
\end{figure}

For the analysis of the model, the equilibrium points of \(S\) and \(I\) were determined from the system of equations. This creates damped oscillations in the dynamics of the infected population at the initial states when graphing infection trajectories at the initial states. 

Solving for the susceptible equilibrium by setting \(\frac{dI}{dt} = 0\):

    \begin{align*}
      \frac{dI}{dt} &= \frac{\beta}{N} SI - \gamma I - \mu I \\ 
      0 &= \frac{\beta}{N} SI - \gamma I - \mu I \\ 
      0 &= I( \frac{\beta}{N} S - \gamma - \mu ) \\ 
       \frac{\beta}{N} S &=  \gamma + \mu\\ 
       S &=  \frac{(\gamma + \mu)N}{\beta} \\
    \end{align*}
    
Substituting S into \(\frac{dS}{dt} = 0\) to solve for the infectious equilibrium:
	
	\begin{align*}
	\frac{dS}{dt} &= \mu N -\frac{\beta}{N} SI - \mu S \\ 
	0 &= \mu N -\frac{\beta}{N} (\frac{(\gamma + \mu)N}{\beta} )I - \mu (\frac{(\gamma + \mu)N}{\beta} ) \\ 
	(\gamma + \mu)I  &= \mu N - \mu (\frac{(\gamma + \mu)N}{\beta} )\\ 
	I &= \frac{\mu N}{\gamma + \mu} (1 - \frac{\gamma + \mu}{\beta}) \\
	\end{align*}
	
\indent Using the measles parameters and initializing the system near the equilibrium points, we derive the following plot of infected individuals over time:

\begin{figure}[H]
  \centering
  \includegraphics[width=0.8\textwidth]{Images/Deterministic_Graph.png}
   \caption{Deterministic Model} 
\label{fig:example}
\end{figure}

\indent In the deterministic model, with measles parameters, the infectious population exhibits periodic fluctuations influenced by seasonal forcing. Measles epidemics typically follow cycles of high incidence, followed by periods of low incidence, with outbreaks occurring every 1-3 years due to immunity dynamics and population turnover. To account for the inherent randomness in population dynamics, we construct a stochastic model using the Gillespie algorithm. This model captures the persistent oscillations observed in real-world measles outbreaks. In the stochastic model, we maintain the same parameters but introduce varying vaccination proportions to assess their impact on the epidemic dynamics, comparing the results to the deterministic model over time. 
    
      

%%%%%%%%%%%%%%%%%%%%%%%%%%%%%%%%%%%
%% Results %%
%%%%%%%%%%%%%%%%%%%%%%%%%%%%%%%%%%%
\section{Results}

%%%%%%%%%%%%%%%%%%%%%%%%%%%%%%%%%%%
%% Discussion%%
%%%%%%%%%%%%%%%%%%%%%%%%%%%%%%%%%%%
\section{Discussion}

This is really important stuff.

\bibliographystyle{vancouver}
\bibliography{Acute Triangle}

\end{document}
